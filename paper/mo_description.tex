\section{DEMO: Differential Evolution for Multi-Objective Optimization}

The DEMO (Differential Evolution for Multi-Objective, \citet{demo}) algorithm is used for the multi-objective optimization of acquisition functions, we briefly introduce the algorithm in this section. It should be noted that other multi-objective optimization algorithms can also be used for the proposed MACE algorithm. 

The DEMO algorithm follows the basic procedure of evolutionary algorithms. Firstly, a set of points are randomly sampled to form the initial population, at each iteration, random variations are added to parent population via mutation and crossover to generate the children population, the parent population and children population are compared to create the parent population for the next generation. During the evolution, the Pareto front is recorded. The DEMO algorithm is summarized in Algorithm~\ref{alg:DEMO}.

\begin{algorithm}[h]
    \caption{DEMO}
    \label{alg:DEMO}
    \begin{algorithmic}[1]
        \STATE Randomly sample the $N$ points in the design space to form the initial population $P^1$
        \FOR{t = 1, 2, \dots $G$~}
            \STATE $M^t$     = \texttt{mutation}($P^t$)
            \STATE $C^t$     = \texttt{crossover}($P^t$, $M^t$)
            \STATE $P^{t+1}$ = \texttt{selection}($P^t$, $C^t$)
        \ENDFOR
        \STATE Return the recorded Pareto front
    \end{algorithmic}
\end{algorithm}

% XXX: mutation
At the $i$-th iteration, we denote the population as $P^i \in R^{N \times D}$, where $N$ is the number of population, and $D$ is the dimension of input variables. The mutated population $M^i \in R^{N \times D}$ is generated by 
\begin{equation}
    \label{eq:DEMO_mutation}
    M^i_j = P^i_{r1} + F \times (P^i_{r2} - P^i_{r3}),~j \in \{1 \dots NP\}
\end{equation}
where $M^i_j$ means the $j$-th row of $M^i$, while $P^i_{r1}$, $P^i_{r2}$ and $P^i_{r3}$ are the $\mathit{r1}$-th, $\mathit{r2}$-th and $\mathit{r3}$-th rows of $P^i$, the $r1$, $r2$ and $r3$ are three integers randomly chosen from $[1, N]$. The scaling factor $F \in (0, 1)$ is an algorithm parameter to control the mutation.

After the mutation, crossover operations are performed to generated the children population $C^i \in R^{N \times D}$. For each element of $C^i$, two random numbers $r_r$ and $r_i$ are generated, $r_r$ is a real-valued number uniformly sampled from $(0, 1)$, $r_i$ is an integer number uniformly sampled from $[1, D]$, the element of $C^i$ is calculated by
\begin{equation}
    \label{eq:DEMO_crossover}
    C^i_{jk} = \left\{
        \begin{array}{lll}
            P^i_{jk} & & r_r < \mathit{CR} \text{ and } r_i \neq k \\
            C^i_{jk} & & \text{otherwise}.
        \end{array}
    \right.
\end{equation}
where the crossover rate $\mathit{CR}$ is an algorithm parameter to control the crossover.

Now that we have the parent population $P^i$ and the children population $C^i$,
we perform selection to generate the new population $P^{i+1}$. Firstly, an
empty archive is created, we perform pair-wise compairsion between parents and
children, if one parent solution dominates its child solution, the parent
solution is added into the archive; if the child solution dominates its parent,
the child is added into the archive; if the parent and its child don't dominate
each other, both the parent and the child are added into the archive. After
the pair-wise compairsion, non-dominated sorting\cite{nsgaii} is
performed to select $G$ solutions as the parent population of the next
generation. The non-dominated sorting method defines a complete order between a
group of solutions, details of the non-dominated sorting can be seen in
\cite{nsgaii}.

As has been mentioned, four algorithm parameters are to be set for the DEMO algorithm: the population size $N$, the number of generations $G$, the scaling factor $F$ and the crossover rate $\mathit{CR}$. We set $N = 100$, $G = 250$, $F = 0.5$ and $\mathit{CR} = 0.3$ for all the experiments performed in the paper. 

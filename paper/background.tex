\section{Background}

% TODO: See how other people describe GP and BO}

\subsection{Problem Formulation}

% TODO: The circuit performance comes from \bf{circuit} simulation}
% TODO: format, citation of HSPICE/spectre}
% TODO: See how other analog circuit optimization paper formulate the problem}

We handle the scenarios where topology of the analog circuit is fixed, this is
practical as thare are usually a lot of classical topologies for given design
task. Once the circuit topology is fixed, the designer has to choose the
appropriate design parameters according to the specifications and the circuit
device model. What we want to do is to automatically search for the optimal
design parameters, the problem can then be formulated as a bound-constrained
black-box optimization problem:

\begin{equation}
    \label{eq:Formulation}
    \text{minimize}~\mathrm{FOM}(\bm{x})
\end{equation}

where $\bm{x} \in \textrm{D} \subset \textrm{R}^d$ is the design variables
and $f_i(\bm{x})$ represents the $i$-th cared circuit performance like gain
or phase margin of an amplifier. and the $\mathrm{FOM}(\bm{x})$ is the
design objective. Given the design parameters $\bm{x}$, the FOM value can be
obtained via circuit simulation using softwares like HSPICE or spectre.

% TODO: Explain the noise

\subsection{Gaussian Process Regression}

We model the $\mathrm{FOM}(\bm{x})$ in \eqref{eq:Formulation} with Gaussian
process (GP) model~\cite{GPML}. The GP model is the most commonly used model
for Bayesian optimization, the advantage of GP is that it can give a
well-calibrated uncertainty of prediction. GP is characterized by a mean
function $m(\bm{x})$ and a covariance function $k(\bm{x}, \bm{x'})$. In this
paper, we use squared-exponential ARD kernel, and a constant mean function
$m(\bm{x}) = \mu_0$ for all our experiments.

Given a new data point $\bm{x}$, the prediction of $f(\bm{x})$ is
not a scalar value, but a predictive distribution 
\begin{equation}
f(\bm{x}) \sim N(\mu(\bm{x}),
\sigma^2(\bm{x}))
\label{eq:GPRPred}
\end{equation}
where $\mu(\bm{x})$ and $\sigma^2(\bm{x})$ can be expressed as
\begin{equation}
        \begin{array}{lll}
            \mu(\bm{x}) &=& \mu_0 + k(\bm{x},X)K^{-1}(\bm{y} - \mu_0) \\
            \sigma^2(\bm{x}) &=& \sigma_f^2 - k(\bm{x}, X)K^{-1}k(X, \bm{x})
        \end{array}
    \label{eq:GPRPredEqNoisy}
\end{equation}
where $k(\bm{x}, X) = (k(\bm{x}, \bm{x}_1), \dots, k(\bm{x},
\bm{x}_N))^T$ and $k(X, \bm{x}) = k(\bm{x}, X)^T$. The
$\mu(\bm{x})$ can be viewed as the prediction of the function value, while
the $\sigma^2(\bm{x})$ is a measure of uncertainty. 

\subsection{Bayesian Optimization}

Bayesian optimization~\cite{shahriari2016taking} was proposed for the
optimization of expensive black-box functions. It consists of two essential
ingredients, i.e., the probabilistic surrogate models and an acquisition
function. The probabilistic surrogate models provide the prediction with
uncertainties. They are refined incrementally with newly observed data.
Acquisition function is used to explore the state space based on the surrogate
model optimally. The procedure of Bayesian optimization is described in
Algorithm~\ref{alg:BayesianOptAlgo}.

\begin{algorithm}
\caption{Bayesian Optimization}
\label{alg:BayesianOptAlgo}
\begin{algorithmic}[1]
\STATE Initial Sampling
\STATE Construct initial GP model
\FOR{t = 1, 2, \dots}
\STATE Construct the acquisition function
\STATE Find $\bm{x}_t$ that optimizes the acquisition function
\STATE Sample $y_t = f(\bm{x}_t)$
\STATE Update probabilistic surrogate model
\ENDFOR
\STATE Return best $f(\bm{x})$ recorded during iterations
\end{algorithmic}
\end{algorithm}

In Bayesian optimization described in Algorithm~\ref{alg:BayesianOptAlgo}, the acquisition function is used to balance the exploration and exploitation during the optimization. The acquisition function considers both the predictive value and the uncertainty. There are a lot of existing acquisition functions, examples include the lower confidence bound (LCB), the probability of improvement (PI), and the expected improvement (EI).

The LCB function is defined as follows:
\begin{equation}
    \label{eq:LCB}
    \mathrm{LCB}(\bm{x}) = \mu(\bm{x}) - \kappa \sigma(\bm{x})
\end{equation}
Where the $\mu(\bm{x})$ and the $\sigma(\bm{x})$ are the predictive value and uncertainty of GP. $\kappa$ is a parameter that balances the exploitation and exploration. 

Following the suggestion of~\cite{brochu2010tutorial}, the $\kappa$ in \eqref{eq:LCB} is defined as:
\begin{equation}
    \label{eq:LCB}
    \begin{array}{lll}
        \kappa &=& \sqrt{\nu \tau_t} \\
        \tau_t &=& 2 \log(t^{d/2+2} \pi^2 / 3 \delta)
    \end{array}
\end{equation}
Where $t$ is the number of current iteration. $\nu$ and $\delta$ are two user defined parameters, we fix $\nu = 0.5$ and $\delta = 0.05$ in this paper for the proposed MACE algorithm and our implementation of the BLCB algorithm.

The PI and EI function are defined as
\begin{equation}
    \label{eq:PI_EI}
    \begin{array}{lll}
        \mathrm{PI}(\bm{x}) &=& \Phi(\lambda) \\
        \mathrm{EI}(\bm{x}) &=& \sigma(\bm{x}) (\lambda \Phi(\lambda) + \phi(\lambda))     \\
        \mathrm{\lambda}    &=& \displaystyle \frac{\tau - \xi - \mu(\bm{x})}{\sigma(\bm{x})}, 
    \end{array}
\end{equation}
where $\tau$ is the current best value objective value, and $\xi$ is a small positive jitter to improvement the ability of exploration. The $\Phi(.)$ and $\phi(.)$ functions are the CDF and PDF functions of normal distribution.

% TODO: discuss the different behaviours of LCB, EI, PI

There are also other acquisition function, like the knowledge gradient~\cite{scott2011correlated} function, predictive entropy search~\cite{hernandez2014predictive}. A portforlio of several acquisition functions is also possible~\cite{hoffman2011portfolio}.

\section{Proposed Batched Bayesian Optimization Algorithm}

\subsection{Multi-objective Optimization}\label{sec:MOForumlation}

\textcolor{red}{See how other MOBO paper formulate multi-objective optimization}


\textcolor{red}{ALERT: COPIED FROM MY TCAS-I paper}

When we have more than one objective to optimize, the problem is formulated as:
\begin{equation}
    \label{eq:MOFormulation}
    \begin{aligned}
        & \text{minimize} & & f_1(\bm{x}),~\dots~,f_m(\bm{x})
    \end{aligned}
\end{equation}

Unlike the single-objective problem, there is usually no \emph{best} design for
a multi-objective problem, as objectives can be conflicting, and it is unlikely
to optimize all of them simultaneously. The goal of a multi-objective
optimization algorithm is to find a set of solutions to approximate the
$\emph{Pareto front}$. A solution $\bf{x}_1$ is said to \emph{dominate} a
solution $\bf{x}_2$ if $\forall i \in \{1 \dots m\},~f_i(\bf{x}_1) \le
f_i(\bf{x}_2)$ and $\exists j \in \{1\dots m\}, f_j(\bf{x}_1) < f_j(\bf{x}_2)$.
A point $\bf{x}_*$ is $\emph{weakly Pareto-optimal}$ if $\forall \bf{x} \in
\bf{\textrm{X}}$, $\bf{x}_*$ is not dominated by $\bf{x}$.  A point $\bf{x}_*$
is $\emph{Pareto-optimal}$ if $\forall x \in \bf{\textrm{X}}$, $\bf{x}_*$ is
not dominated by $\bf{x}$ and $\exists i \in \{1 \dots m\},~f_i(\bf{x}_*) <
f_i(\bf{x})$. The set of Pareto optimal outcomes is called the \emph{Pareto
front}.

\subsection{MACE}

% TODO: discuss the difference of acquisition functions: LCB: do not avoid repeated sampling, EI: always positive

\textcolor{red}{Plot a figure to illustrate the PF/PS of acquisition functions}

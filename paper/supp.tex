%%%%%%%% ICML 2018 EXAMPLE LATEX SUBMISSION FILE %%%%%%%%%%%%%%%%%

\documentclass{article}

% Recommended, but optional, packages for figures and better typesetting:
\usepackage{microtype}
\usepackage{graphicx}
\usepackage{subfigure}
\usepackage{booktabs} % for professional tables

% Added by wenlong lyu
\usepackage{amsmath}
\usepackage[plain]{fancyref}
\usepackage{bm}
\usepackage{todonotes}
\usepackage{placeins}
\usepackage{multirow}

% hyperref makes hyperlinks in the resulting PDF.
% If your build breaks (sometimes temporarily if a hyperlink spans a page)
% please comment out the following usepackage line and replace
% \usepackage{icml2018} with \usepackage[nohyperref]{icml2018} above.
\usepackage{hyperref}

% Attempt to make hyperref and algorithmic work together better:
\newcommand{\theHalgorithm}{\arabic{algorithm}}

% % Use the following line for the initial blind version submitted for review:
% \usepackage{icml2018}

% If accepted, instead use the following line for the camera-ready submission:
\usepackage[accepted]{icml2018}

% The \icmltitle you define below is probably too long as a header.
% Therefore, a short form for the running title is supplied here:
\icmltitlerunning{Supplementary Materials}

\begin{document}

\twocolumn[
\icmltitle{Supplementary Materials}

% It is OKAY to include author information, even for blind
% submissions: the style file will automatically remove it for you
% unless you've provided the [accepted] option to the icml2018
% package.

% List of affiliations: The first argument should be a (short)
% identifier you will use later to specify author affiliations
% Academic affiliations should list Department, University, City, Region, Country
% Industry affiliations should list Company, City, Region, Country

% You can specify symbols, otherwise they are numbered in order.
% Ideally, you should not use this facility. Affiliations will be numbered
% in order of appearance and this is the preferred way.
% \icmlsetsymbol{equal}{*}

% \begin{icmlauthorlist}
%     \icmlauthor{Wenlong Lyu}{fudan}
%     \icmlauthor{Fan Yang}{fudan}
%     \icmlauthor{Changhao Yan}{fudan}
%     \icmlauthor{Dian Zhou}{fudan,utd}
%     \icmlauthor{Xuan Zeng}{fudan}
% % \icmlauthor{Aeiau Zzzz}{equal,to}
% % \icmlauthor{Bauiu C.~Yyyy}{equal,to,goo}
% % \icmlauthor{Cieua Vvvvv}{goo}
% % \icmlauthor{Iaesut Saoeu}{ed}
% % \icmlauthor{Fiuea Rrrr}{to}
% % \icmlauthor{Tateu H.~Yasehe}{ed,to,goo}
% % \icmlauthor{Aaoeu Iasoh}{goo}
% % \icmlauthor{Buiui Eueu}{ed}
% % \icmlauthor{Aeuia Zzzz}{ed}
% % \icmlauthor{Bieea C.~Yyyy}{to,goo}
% % \icmlauthor{Teoau Xxxx}{ed}
% % \icmlauthor{Eee Pppp}{ed}
% \end{icmlauthorlist}

% % \icmlaffiliation{utd}{utdallas}
% \icmlaffiliation{fudan}{State Key Lab of ASIC and System, Microelectronics Department, Fudan University, Shanghai, China}
% \icmlaffiliation{utd}{Department of Electrical Engineering, University of Texas at Dallas, Richardson, TX, U.S.A}
% % \icmlaffiliation{to}{Department of Computation, University of Torontoland, Torontoland, Canada}
% % \icmlaffiliation{goo}{Googol ShallowMind, New London, Michigan, USA}
% % \icmlaffiliation{ed}{School of Computation, University of Edenborrow, Edenborrow, United Kingdom}

\icmlcorrespondingauthor{Xuan Zeng}{xzeng@fudan.edu.cn}
% \icmlcorrespondingauthor{Eee Pppp}{ep@eden.co.uk}

% You may provide any keywords that you
% find helpful for describing your paper; these are used to populate
% the "keywords" metadata in the PDF but will not be shown in the document
\icmlkeywords{Bayesian Optimization, Automated analog circuit design}

\vskip 0.3in
]

% this must go after the closing bracket ] following \twocolumn[ ...

% This command actually creates the footnote in the first column
% listing the affiliations and the copyright notice.
% The command takes one argument, which is text to display at the start of the footnote.
% The \icmlEqualContribution command is standard text for equal contribution.
% Remove it (just {}) if you do not need this facility.

% XXX:
% Done:
% 3.  Additional experiments with variety of batch sizes
% TODO:
% 1.  Description of the DEMO algorithm
% 2.  Demonstration of the DEMO algorithm
% 4.  The paper contains wrong usages of the cite command, for example in line 30, right side. In that case, \citet should be used instead of \citep. There are several other occasions of this throughout the paper.
% 5.  The paper confuses 'Bayesian optimization' (Shariari et al., 2016) with 'the Bayesian optimization algorithm' (Pelikan et al., http://martinpelikan.net/boa.html) at some locations. Those are actually two different optimization algorithms and the paper should only refer to 'Bayesian optimization'.
% 6.  line 114, left side: the sentence abruptly ends in the middle!
% 7.  line 125: as the experiments only use m(x) = 0, I think it's not necessary to include the mean function into the GP prediction equation.
% 8.  Section 2.3, the first two paragraphs are not written very well and I suggest the authors to rewrite them more clearly.
% 9.  Equation 5 is originally from Srinivas et al. (2010: Gaussian process optimization in the bandit setting: No regret and experimental design. It would be good if the authors reference original literature where possible.
% 10. The points denoting the Pareto set in Figure 1 are not really readable when printing the paper. I suggest that the authors increase the size of the markers and change the colors.
% 12. The sentence in line 218 (left) makes no sense, LCB is supposed to be minimized, while EI should be maximized.
% 13. Overhead of the pareto optimization algorithm DEMO.
% 14. It would be great if the caption of Table 2 mentions that this is the regret.
% 15. The paper does not mention the parallelization mentioned in Snoek et al. (2012). It would also be great if the authors could compare against Spearmint with four workers.
\section{DEMO: Differential Evolution for Multi-Objective Optimization}

The DEMO (Differential Evolution for Multi-Objective, \citet{demo}) algorithm is used for the multi-objective optimization of acquisition functions, we briefly introduce the algorithm in this section. It should be noted that other multi-objective optimization algorithms can also be used for the proposed MACE algorithm. 

The DEMO algorithm follows the basic procedure of evolutionary algorithms. Firstly, a set of points are randomly sampled to form the initial population, at each iteration, random variations are added to parent population via mutation and crossover to generate the children population, the parent population and children population are compared to create the parent population for the next generation. During the evolution, the Pareto front is recorded. The DEMO algorithm is summarized in Algorithm~\ref{alg:DEMO}.

\begin{algorithm}[h]
    \caption{DEMO}
    \label{alg:DEMO}
    \begin{algorithmic}[1]
        \STATE Randomly sample the $N$ points in the design space to form the initial population $P^1$
        \FOR{t = 1, 2, \dots $G$~}
            \STATE $M^t$     = \texttt{mutation}($P^t$)
            \STATE $C^t$     = \texttt{crossover}($P^t$, $M^t$)
            \STATE $P^{t+1}$ = \texttt{selection}($P^t$, $C^t$)
        \ENDFOR
        \STATE Return the recorded Pareto front
    \end{algorithmic}
\end{algorithm}

% XXX: mutation
At the $i$-th iteration, we denote the population as $P^i \in R^{N \times D}$, where $N$ is the number of population, and $D$ is the dimension of input variables. The mutated population $M^i \in R^{N \times D}$ is generated by 
\begin{equation}
    \label{eq:DEMO_mutation}
    M^i_j = P^i_{r1} + F \times (P^i_{r2} - P^i_{r3}),~j \in \{1 \dots NP\}
\end{equation}
where $M^i_j$ means the $j$-th row of $M^i$, while $P^i_{r1}$, $P^i_{r2}$ and $P^i_{r3}$ are the $\mathit{r1}$-th, $\mathit{r2}$-th and $\mathit{r3}$-th rows of $P^i$, the $r1$, $r2$ and $r3$ are three integers randomly chosen from $[1, N]$. The scaling factor $F \in (0, 1)$ is an algorithm parameter to control the mutation.

After the mutation, crossover operations are performed to generated the children population $C^i \in R^{N \times D}$. For each element of $C^i$, two random numbers $r_r$ and $r_i$ are generated, $r_r$ is a real-valued number uniformly sampled from $(0, 1)$, $r_i$ is an integer number uniformly sampled from $[1, D]$, the element of $C^i$ is calculated by
\begin{equation}
    \label{eq:DEMO_crossover}
    C^i_{jk} = \left\{
        \begin{array}{lll}
            P^i_{jk} & & r_r < \mathit{CR} \text{ and } r_i \neq k \\
            C^i_{jk} & & \text{otherwise}.
        \end{array}
    \right.
\end{equation}
where the crossover rate $\mathit{CR}$ is an algorithm parameter to control the crossover.

Now that we have the parent population $P^i$ and the children population $C^i$,
we perform selection to generate the new population $P^{i+1}$. Firstly, an
empty archive is created, we perform pair-wise compairsion between parents and
children, if one parent solution dominates its child solution, the parent
solution is added into the archive; if the child solution dominates its parent,
the child is added into the archive; if the parent and its child don't dominate
each other, both the parent and the child are added into the archive. After
the pair-wise compairsion, non-dominated sorting\cite{nsgaii} is
performed to select $G$ solutions as the parent population of the next
generation. The non-dominated sorting method defines a complete order between a
group of solutions, details of the non-dominated sorting can be seen in
\cite{nsgaii}.

As has been mentioned, four algorithm parameters are to be set for the DEMO algorithm: the population size $N$, the number of generations $G$, the scaling factor $F$ and the crossover rate $\mathit{CR}$. We set $N = 100$, $G = 250$, $F = 0.5$ and $\mathit{CR} = 0.3$ for all the experiments performed in the paper. 

\section{Additional Experiments with Varied Batch Sizes}

We performed additional experiments with varied batch sizes $B = 2$, $B = 3$
and $B = 5$, the results for the analytical benchmark functions are shown in \Fref{tab:result_analytical_b2},
\Fref{tab:result_analytical_b3} and \Fref{tab:result_analytical_b5}. The
optimization results of the operational amplifier are listed in
\Fref{tab:result_opamp_vary_B}, the optimization results of the class-E power
amplifier are given in \Fref{tab:result_classE_vary_B}. With
varied batch sizes, the proposed MACE method remain competitive compared with
the state-of-the-art batch Bayesian optimization methods.

\begin{table*}[!htb]
    \centering
    \caption{Statistics of the regrets of the benchmark functions with batch size $B=2$}
    \label{tab:result_analytical_b2}
    \begin{tabular}{lllllll}
        \toprule
        Algorithm     & MACE                                   & BLCB                       & EI-LP                        & QKG                    & QEI                              \\ 
        \midrule
        Ackley        & 1.15             $\pm$  0.646    &  1.7       $\pm$  0.85      &  \textbf{0.507  $\pm$ 0.408}   &  4.31   $\pm$  1.81    &  3.07      $\pm$  0.786    \\
        Alpine1       & \textbf{1.38     $\pm$  0.768}   &  3.23      $\pm$  1.03      &  1.55           $\pm$ 0.689    &  3.17   $\pm$  0.749   &  2.4       $\pm$  0.904    \\
        Branin        & \textbf{5.6e-6   $\pm$  1.03e-5} &  1.86e-4   $\pm$  2.65e-4   &  0.0257         $\pm$ 0.0395   &  0.21   $\pm$  0.159   &  8.27e-4   $\pm$  1.56e-3  \\
        Eggholder     & 116              $\pm$  65.4     &  132       $\pm$  66.2      &  \textbf{82.7   $\pm$ 51.6}    &  115    $\pm$  78.3    &  104       $\pm$  78.4     \\
        Hartmann6     & \textbf{0.0479   $\pm$  0.0584}  &  0.0719    $\pm$  0.0587    &  0.161          $\pm$ 0.301    &  0.257  $\pm$  0.0823  &  0.178     $\pm$  0.128    \\
        Rosenbrock    & \textbf{1.05e-3  $\pm$  0.0011}  &  5.56e-3   $\pm$  9.28e-3   &  8.37           $\pm$ 5.63     &  9.41   $\pm$  10.7    &  10.3      $\pm$  8.52     \\
        Ackley10D     & \textbf{2.75     $\pm$  0.497}   &  3.13      $\pm$  0.723     &  18.5           $\pm$ 1.02     &  18.4   $\pm$  0.943   &  18.8      $\pm$  0.608    \\
        Rosenbrock10D & \textbf{223      $\pm$  104}     &  552       $\pm$  223       &  1.1e+03        $\pm$ 496      &  957    $\pm$  439     &  757       $\pm$  405      \\
        \bottomrule
    \end{tabular}
\end{table*}

\begin{table*}[!htb]
    \centering
    \caption{Statistics of the regrets of the benchmark functions with batch size $B=3$}
    \label{tab:result_analytical_b3}
    \begin{tabular}{lllllll}
        \toprule
        Algorithm     & MACE                             & BLCB                       & EI-LP                        & QKG                    & QEI                              \\ 
        \midrule
        Ackley        & 1.37            $\pm$  1.39      &  1.71         $\pm$  1.02    &  \textbf{0.216 $\pm$  0.148}  &  5.49   $\pm$  1.94   &  2.34      $\pm$  0.788     \\
        Alpine1       & \textbf{1.03    $\pm$  0.746}    &  2.63         $\pm$  1.2     &  1.1           $\pm$  0.376   &  3.18   $\pm$  0.225  &  2.25      $\pm$  0.42      \\
        Branin        & \textbf{2.85e-5 $\pm$  3.18e-5}  &  8.14e-5      $\pm$  1.27e-4 &  0.0344        $\pm$  0.0183  &  0.247  $\pm$  0.188  &  5.21e-5   $\pm$  1.35e-4   \\
        Eggholder     & \textbf{65.3    $\pm$  62.9}     &  82.6         $\pm$  32.2    &  65.9          $\pm$  43.3    &  117    $\pm$  79.2   &  81.7      $\pm$  63.1      \\
        Hartmann6     & \textbf{0.012   $\pm$  0.0359}   &  0.0477       $\pm$  0.0584  &  0.0489        $\pm$  0.0531  &  0.335  $\pm$  0.188  &  0.189     $\pm$  0.108     \\
        Rosenbrock    & \textbf{9.46e-4 $\pm$  7.75e-4}  &  0.00148      $\pm$  0.00212 &  3.78          $\pm$  3.4     &  4.28   $\pm$  5.5    &  5.44      $\pm$  4.21      \\
        Ackley10D     & 3.05            $\pm$  0.682     &  \textbf{3.05 $\pm$  0.431}  &  17.6          $\pm$  3.53    &  18.5   $\pm$  0.731  &  18.6      $\pm$  0.438     \\
        Rosenbrock10D & \textbf{208     $\pm$  92.5}     &  389          $\pm$  187     &  653           $\pm$  473     &  695    $\pm$  307    &  953       $\pm$  410       \\
        \bottomrule
    \end{tabular}
\end{table*}

\begin{table*}[!htb]
    \centering
    \caption{Statistics of the regrets of the benchmark functions with batch size $B=5$}
    \label{tab:result_analytical_b5}
    \begin{tabular}{lllllll}
        \toprule
        Algorithm     & MACE                       & BLCB                        & EI-LP                   & QKG                   & QEI                         \\ 
        \midrule
        Ackley        & 1.7             $\pm$  1.02     &  1.38           $\pm$  0.836   &  \textbf{0.105 $\pm$  0.0978} &  5.27  $\pm$  1.38   &  2.16         $\pm$  1.11      \\
        Alpine1       & \textbf{0.654   $\pm$  0.317}   &  1.68           $\pm$  1.26    &  0.766         $\pm$  0.441   &  3.21  $\pm$  0.497  &  2.05         $\pm$  0.341     \\
        Branin        & \textbf{1.26e-5 $\pm$  1.81e-5} &  2.99e-5        $\pm$  3.42e-5 &  0.0144        $\pm$  0.0154  &  0.163 $\pm$  0.163  &  2.02e-5      $\pm$  5.21e-5   \\
        Eggholder     & 74.1            $\pm$  74.3     &  61.1           $\pm$  33.5    &  63.5          $\pm$  94.3    &  71    $\pm$  29.4   &  \textbf{49.1 $\pm$  25.8}      \\
        Hartmann6     & 0.0477          $\pm$  0.0584   &  \textbf{0.0358 $\pm$  0.0546} &  0.0552        $\pm$  0.0546  &  0.47  $\pm$  0.221  &  0.198        $\pm$  0.105     \\
        Rosenbrock    & \textbf{5.48e-4 $\pm$  8.12e-4} &  9.39e-4        $\pm$  6.83e-4 &  2.72          $\pm$  1.97    &  3.42  $\pm$  4.8    &  6.69         $\pm$  5.34      \\
        Ackley10D     & \textbf{2.63    $\pm$  0.486}   &  3.05           $\pm$  0.319   &  15.7          $\pm$  5.69    &  18.1  $\pm$  0.476  &  18.1         $\pm$  0.653     \\
        Rosenbrock10D & \textbf{81.9    $\pm$  22.9}    &  348            $\pm$  83.7    &  645           $\pm$  470     &  893   $\pm$  393    &  705          $\pm$  314       \\
        \bottomrule
    \end{tabular}
\end{table*}

\begin{table*}[!htb]
    \centering
    \caption{Optimization Results of the Operational Amplifier with $B = 2$, $B = 3$ annd $B = 5$}
    \label{tab:result_opamp_vary_B}
    \begin{tabular}{llll}
        \toprule
        Algorithm & MACE                        & BLCB                & EI-LP               \\ \midrule
        B=2       & \emph{-689  $\pm$  4.37}    &  -649  $\pm$  28.8  &  -627  $\pm$  48.5  \\
        B=3       & \emph{-690  $\pm$  0.518}   &  -672  $\pm$  20.4  &  -621  $\pm$  45.2  \\
        B=5       & \emph{-690  $\pm$  0.0251}  &  -684  $\pm$  6.86  &  -626  $\pm$  49    \\




        \bottomrule
    \end{tabular}
\end{table*}


\begin{table*}[!htb]
    \centering
    \caption{Optimization Results of the class-E Power Amplifier with $B = 2$, $B = 3$ annd $B = 5$}
    \label{tab:result_classE_vary_B}
    \begin{tabular}{llll}
        \toprule
        Algorithm     & MACE                       & BLCB                        & EI-LP          \\
        \midrule
        B=2       & \textbf{-4.13  $\pm$  0.207}  &  -4.01  $\pm$  0.208  &  -3.65  $\pm$  0.312  \\
        B=3       & \textbf{-4.45  $\pm$  0.326}  &  -4.17  $\pm$  0.163  &  -3.87  $\pm$  0.306  \\
        B=5       & \textbf{-4.26  $\pm$  0.18 }  &  -4.17  $\pm$  0.111  &  -4.18  $\pm$  0.222  \\
        \bottomrule
    \end{tabular}
\end{table*}


%%%%%%%%%%%%%%%%%%%%%%%%%%%%%%%%%%%%%%%%%%%%%%%%%%%%%%%%%%%%%%%%%%%%%%%%%%%%%%%

\bibliography{ref}
\bibliographystyle{icml2018}

\end{document}


% This document was modified from the file originally made available by
% Pat Langley and Andrea Danyluk for ICML-2K. This version was created
% by Iain Murray in 2018. It was modified from a version from Dan Roy in
% 2017, which was based on a version from Lise Getoor and Tobias
% Scheffer, which was slightly modified from the 2010 version by
% Thorsten Joachims & Johannes Fuernkranz, slightly modified from the
% 2009 version by Kiri Wagstaff and Sam Roweis's 2008 version, which is
% slightly modified from Prasad Tadepalli's 2007 version which is a
% lightly changed version of the previous year's version by Andrew
% Moore, which was in turn edited from those of Kristian Kersting and
% Codrina Lauth. Alex Smola contributed to the algorithmic style files.

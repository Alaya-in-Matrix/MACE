\section{Introduction}

% % TODO: more introduction to the importance of anlog IC sizing, as the reviewers may not have much knowledge of circuit design
%
The advancement of modern society is driven by the development of Integrated
Circuits (IC). Unlike the digital circuits where the design flow is already
highly automated, the automation of analog circuit design is still a
challenging problem.

Traditionally, the design parameters of analog circuits like widths and lengths of
transistors are manually calculated by designers with their experience and the
understanding of the design specifications. However, due to the progress of IC
manufacture technology forecasted by Moore's law, the circuit devices become
more and more complicated, and the parasitic effect of the circuits can no
longer be ignored. On the other hand, the demands for high-performance,
low-power analog circuits are increasing. It is much more difficult to
meet the performance and time-to-market requirements with manual circuit design.
Automated analog circuit design has thus attracted much research interest in
the past decade~\cite{rutenbar2007hierarchical}.

% TODO: Traditional methods using offline model and simulated based methods are
% not good
The analog circuit design automation problems can be formulated as optimization
problems. The aim is to find the optimal design parameters that provide the
best circuit performance, which can be represented by a figure of merit (FOM)
real-valued function. Prior works about analog circuit optimization
include offline model-based approaches
~\cite{colleran2003optimization,daems2003simulation,wang2014enabling} and
simulation-based approaches. The offline model-based methods try to build
global models of the FOM via manual calculation or regression with simulated
data and then optimize the cheap-to-evaluate models. The problem with this
approach is that the accurate models are usually hard to get. For example,
in~\citet{wang2014enabling}, 100,000 randomly simulated points are used to train
a sparse polynomial model for an amplifier circuit with ten design parameters.

Simulation-based methods, instead, treat the performances of the circuits as black-box functions. The performances are obtained from circuit simulations. Global optimization
algorithms are directly applied to the black-box functions. For
simulation-based circuit optimization methods, meta-heuristic
algorithms~\cite{phelps2000anaconda, liu2009analog} are widely used. Although
these algorithms can explore the whole design space, they have relatively low
convergence rate. When the circuit simulation takes a long time, both
model-based and simulation-based approaches can be very time-consuming.

% TODO: Bayesian optimization is a sequential algorithm, there is a need to
% parallelize it

In recent years, the Gaussian process (GP)~\cite{GPML} model has been
introduced for the automated design of analog circuits to reduce the
required number of circuit simulations. In~\citet{liu2014gaspad}, GP is combined
with differential evolution algorithm. Recently, the Bayesian optimization
(BO)~\cite{shahriari2016taking} algorithm has also been applied for analog
circuit optimization. In~\citet{lyu2017efficient}, the Bayesian optimization
algorithm is firstly introduced for the single- and multi-objective
optimization of general analog circuits and has shown to be much more efficient
compared with other simulation-based approaches. In~\citet{wang2017efficient},
the Bayesian optimization algorithm is combined with adaptive Monte-Carlo
sampling to optimize the yield of analog circuits and static random-access
memory (SRAM).

The Bayesian optimization algorithm is a well-studied algorithm and has
demonstrated to be promising for the automated design of analog circuits.
However, the standard Bayesian optimization algorithm is sequential. It chooses
only one point at each iteration by optimizing the acquisition function. It is
often desirable to select a batch of points at each iteration. The sequential
property of Bayesian optimization limits its further applications in multi-core computer systems.

The Bayesian optimization algorithm has been extended to enable batch
selection. Some prior works, like the qEI \cite{qEI}, qKG \cite{wu2016parallel}
and parallel predictive entropy search (PPES)~\cite{shah2015parallel}
approaches, consider to search for the optimal batch selection for a specific
acquisition function. These methods usually involve some approximations or
Monte-Carlo sampling, and thus scale poorly as the batch size increases. Other
works, including the simulation matching (SM)~\cite{azimi2010batch} method, the
batch-UCB (BUCB, BLCB for minimization
problems)~\cite{desautels2014parallelizing} method, the parallel UCB with pure
exploration (GP-UCB-PE)~\cite{contal2013parallel} method, and the local
penalization (LP)~\cite{gonzalez2016batch} method, adopted the \emph{greedy}
strategies that select individual points until the batch is filled.

% TODO: Beiefly introduce my algorithm
All the batch Bayesian optimization algorithms mentioned above choose to use single acquisition function.
And except for the SM method~\cite{azimi2010batch} and LP method~\cite{gonzalez2016batch} which can use arbitrary acquisition
function, other parallelization methods rely on a specific acquisition
function. The UCB acquisition function must be used for BUCB and GP-UCB-PE, and
the knowledge gradient (KG) acquisition function must be used for the qKG algorithm. As is stated
in~\citet{hoffman2011portfolio}, no single acquisition function can always
outperform other acquisition functions. Relying on one acquisition function may
result in poor performance.

In this paper, we propose to parallelize the Bayesian optimization algorithm
via the Multi-objective ACquisition Ensemble (MACE). The proposed MACE method
exploits the disagreement between different acquisition functions to enable
batch selection. At each iteration, after the GP model is updated, multiple
acquisition functions are selected. We then perform multi-objective
optimization to find the \emph{Pareto front} (PF) of the acquisition functions.
The PF represents the best trade-off between these acquisition functions. When
batch evaluations are possible, we can sample multiple points on the PF to accelerate the optimization.

The MACE algorithm is tested using several analytical benchmark functions and
two real-world analog circuits, including an operational amplifier with ten
design parameters and a class-E power amplifier with twelve design parameters.
The BLCB method~\cite{desautels2014parallelizing}, local penalization method with expected improvement
acquisition function (EI-LP)~\cite{gonzalez2016batch}, qEI \cite{qEI} and qKG \cite{wu2016parallel} methods are compared with
MACE. The proposed MACE method achieved competitive performance when compared
with the state-of-the-art algorithms listed in the paper.

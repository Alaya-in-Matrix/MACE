\section{Introduction}

% % TODO: more introduction to the importance of anlog IC sizing, as the reviewers may not have much knowledge of circuit design
Automated analog circuit design has been a challenging problem for the
community of electronic design automation (EDA). Unlike digital circuits, where
the design flow is highly automated, analog circuit design still
heavily relies on designer's experience. The design parameters of analog circuits like
transistor widths and lengths need to be manually calculated based on the
specifications and the designers' understanding of the circuit. However, due to
the ever-scaling IC manufacture technology and the increasing demands for
high-performance, low-power circuits, it is getting much more difficult to meet
the performance and time-to-market requirement with manual circuit design.
Automated analog circuit design has thus attracted more research interests in
the past decade~\cite{rutenbar2007hierarchical}.

% TODO: Traditional methods using offline model and simulated based methods are
% not good
The analog circuit design automation problems can be formulated as optimization
problems, the aim is to find the optimal vector of design parameters that give
best figure of merit (FOM). Prior works about analog circuit optimization
include offline model-based approaches
~\cite{colleran2003optimization,daems2003simulation,wang2014enabling} and
simulation based approaches. The offline model-based methods try to build
accurate global model and then apply global optimization algorithms to the
cheap-to-evaluate models, the model either comes from designers' manual
derivation or from regression models like SVM and ANN. The problem of this
approach is that the accurate models are usually hard to get, for example,
in~\cite{wang2014enabling}, 100,000 randomly simulated points are used to train
a sparse polynomial model for a circuit with ten design parameters.
Simulation-based methods, instead, treat the circuits as black-box functions
that get the objective value from circuit simulations, global optimization
algorithms are directly applied to the black-box functions. For
simulation-based circuit optimization methods, meta-heuristic algorithms like
particle swarm intelligence (PSO)~\cite{phelps2000anaconda} and differential
evolution algorithm (DE)~\cite{liu2009analog} are widely used, although these
algorithm are able to explore the whole design space, they have relatively low
convergence rate and consume many circuit simulations. When the circuit
simulation takes a long time, both model-based and simulation-based approaches
can be prohibitive.

% TODO: Bayesian optimization is a sequential algorithm, there is a need to
% parallelize it
%
% TODO: cite the two 3D-IC papers of TVLSI
To reduce the number of circuit simulations, the Gaussian process
(GP)~\cite{GPML} model has been introduced as online surrogate model to assist
the optimization. In~\cite{liu2014gaspad}, GP is combined with differential
evolution algorithm, in~\cite{lyu2017efficient}. Recently, the Bayesian
optimization (BO)~\cite{shahriari2016taking} algorithm has been applied for
analog circuit optimization, the Bayesian optimization algorithm is firstly
introduced for the automated design of general analog circuits, and has shown
very good performance compared to other simulation-based approaches,
in~\cite{wang2017efficient}, the Bayesian optimization algorithm is combined
with adaptive Monte-Carlo sampling to optimize the yield of analog circuits and
static random-access memory (SRAM).

The Bayesian optimization algorithm is a well-studied algorithm, and has
demonstrated to be a promising algorithm for automated analog circuit design,
however, the standard Bayesian optimization algorithm is sequential, it chooses
the next evaluation point by optimizing the specified acquisition fucntion,
which makes the parallelization of Bayesian optimization non-trivial. The
sequential property of BO limits its further application, as multi-core
processors are usually availble for modern computers. 

% TODO: Review current methods
% TODO: See how other papers review existing batched BO methods
Batched Bayesian optimization has been a \textcolor{red}{hot} topic in the BO
literature. Related works include the \emph{simulation matching}
method~\cite{azimi2010batch}, the \emph{BUCB} (BLCB for minimization)
method~\cite{desautels2014parallelizing}, the \emph{GP-UCB-PE}
method~\cite{contal2013parallel}, the \emph{local penalization}
method~\cite{gonzalez2016batch}, the \emph{parallel predictive entropy
search}~\cite{shah2015parallel} and the \emph{qKG}
method~\cite{wu2016parallel}.

% BUCB, UCB-PE: regret bound
% PPES: entropy search
% qKG, qEI: optimal decision if multiple points are selected
% LP: very good heuristic

% TODO: Beiefly introduce my algorithm
All the above mentioned algorithms choose to use one acquisition function, and
except for the smulation mathcing method and local penalization method which
can select arbitrary acquisition function, other parallelization methods rely on
specific acquisition function, the UCB acquisition fucntion must be used for
BUCB and GP-UCB-PE, and the knowledge gradient method must be used for qKG
algorithm. As has been stated in~\cite{hoffman2011portfolio}, no one
acquisition fucntion can always outperform other acquisition functions, so
relying on one acquisition fucntion may resulting in poor performance. Also, a
portfolio strategy that ensembles multiple acquisition functions would be
helpful.

In this paper, we propose to parallize the Bayesian optimization algorithm via
a multi-objective ensemble of acquisition functions. Firstly, in each
iteration, after the GP model is updated, multiple acquisition functions are
selected, we then perform multi-objective optimization to find the \emph{Pareto
front} (PF) of the acquisition functions. The PF represents the best trade-off
between these acquisition functions, when used in sequential mode, the propose
multi-objective acquisition ensemble (MACE) strategy can be seen as a portfolio
strategy, when batched evaluation is possible, we can sample multiple points on
the PF as there are usually much more points on the PF than the batch size.

We tested the MACE algorithm using several analytical benchmark functions and
two real-world analog circuits, including an operational amplifier with
\textcolor{red}{14} design parameters and a class-E power amplifier with
\textcolor{red}{ten} design parameters. The BLCB method, local penalization
method with expected improvement acquisition fucntion (EI-LP), the qEI methods
and qEI methods are compared with MACE. The proposed MACE method achieved
competitive performance when compared with the listed state-of-the-art
algorithms.

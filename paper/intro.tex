\section{Introduction}

% % TODO: more introduction to the importance of anlog IC sizing, as the reviewers may not have much knowledge of circuit design
Automated analog circuit design has been a challenging problem for the
community of electronic design automation (EDA). Unlike digital circuits, where
the design flow is highly automated, analog circuit design still
heavily relies on designer's experience. The design parameters of analog circuits like
transistor widths and lengths need to be manually calculated based on the
specifications and the designers' understanding of the circuit. However, due to
the ever-scaling IC manufacture technology and the increasing demands for
high-performance, low-power circuits, it is getting much more difficult to meet
the performance and time-to-market requirement with manual circuit design.
Automated analog circuit design has thus attracted more research interests in
the past decade~\cite{rutenbar2007hierarchical}.

% TODO: Traditional methods using offline model and simulated based methods are
% not good
The analog circuit design automation problems can be formulated as optimization
problems, the aim is to find the optimal vector of design parameters that give
best figure of merit (FOM). Prior works about analog circuit optimization
include offline model-based approaches
~\cite{colleran2003optimization,daems2003simulation,wang2014enabling} and
simulation based approaches. The offline model-based methods try to build
accurate global model and then apply global optimization algorithms to the
cheap-to-evaluate models, the model either comes from designers' manual
derivation or from regression models like SVM and ANN. The problem of this
approach is that the accurate models are usually hard to get, for example,
in~\cite{wang2014enabling}, 100,000 randomly simulated points are used to train
a sparse polynomial model for a circuit with ten design parameters.
Simulation-based methods, instead, treat the circuits as black-box functions
that get the objective value from circuit simulations, global optimization
algorithms are directly applied to the black-box functions. For
simulation-based circuit optimization methods, meta-heuristic algorithms like
particle swarm intelligence (PSO)~\cite{phelps2000anaconda} and differential
evolution algorithm (DE)~\cite{liu2009analog} are widely used, although these
algorithm are able to explore the whole design space, they have relatively low
convergence rate and consume many circuit simulations. When the circuit
simulation takes a long time, both model-based and simulation-based approaches
can be prohibitive.

% TODO: Bayesian optimization is a sequential algorithm, there is a need to
% parallelize it
%
% TODO: cite the two 3D-IC papers of TVLSI
To reduce the circuit simulations needed by the optimization, the Gaussian
process (GP)~\cite{GPML} model has been introduced as online surrogate model to
assist the optimization. In~\cite{liu2014gaspad}, GP is combined with
differential evolution algorithm, in~\cite{lyu2017efficient}. Recently, the
Bayesian optimization (BO) algorithm has been applied for analog circuit
optimization, the Bayesian optimization algorithm is firstly introduced for the
automated design of general analog circuits, and has shown very good
performance compared to other simulation-based approaches,
in~\cite{wang2017efficient}, the Bayesian optimization algorithm is combined
with adaptive Monte-Carlo sampling to optimize the yield of analog circuits and
static random-access memory (SRAM).

The Bayesian optimization algorithm is a well-studied algorithm, and has
demonstrated to be a promising algorithm for automated analog circuit design,
however, the standard Bayesian optimization algorithm is sequential, it chooses
the next evaluation point by optimizing the specified acquisition fucntion,
which makes the parallelization of Bayesian optimization non-trivial. The
sequential property of BO limits its further application, as multi-core
processors are usually availble for modern computers. 

% TODO: Review current methods
There are some research about batched Bayesian optimization in the machine
learning community,

% BUCB, UCB-PE: regret bound
% PPES: entropy search
% qKG, qEI: optimal decision if multiple points are selected
% LP: very good heuristic

% TODO: Beiefly introduce my algorithm
All the above mentioned algorithms choose to use one specified acquisition function, however it has been shown in \cite{}

%For simulation-based methods, the optimization is driven directly by the
%circuit simulations. The objective functions and constraints are viewed as
%black-box functions. They are obtained directly from circuit simulations. Since
%the circuit simulations are invoked on-the-fly, the number of required
%simulations would even be much lower than those of model-based approaches. A
%variety of global optimization algorithms, including simulated
%annealing (SA)~\cite{phelps2000anaconda}, particle swarm intelligence (PSO)
%algorithm~\cite{vural2012analog} and evolutionary
%algorithm~\cite{liu2009analog}, have been proposed to better explore the
%solution space. Stochastic characteristics of these algorithms help them to
%avoid getting stuck in the local optimum. However, they also suffer from
%relatively low convergence rate. Recently, gradient based local search with
%multiple starting points (MSP)~\cite{6420988, 1510411, pengboDate}
%have been reported to be very efficient for analog circuit sizing.

%However, the computation-intensive circuit simulation makes the analog circuit
%sizing challenging for large-scale/complicated analog/RF circuits. For
%example, one transient simulation could take several hours for complicated
%circuits like a PLL. The passive components such as inductors and transformers
%are widely used in RF and microwave circuits.  However, these components are
%usually characterized by Electro-Magnetic (EM) simulations, which could take
%even several hours for one configuration.  Furthermore, if the process variations
%are taken into account, PVT corner enumerations or Monte-Carlo yield
%estimations are required.  The computational cost would thus be greatly
%increased. To address this problem, a hybrid methodology which combines both
%the model-based approach and simulation-based approach has been proposed
%recently~\cite{liu2011synthesis, 5763181, 6218230, tugui2013design,
%liu2014gaspad}. Instead of using pre-built offline models, online evolving
%models are built. The circuit simulations are invoked during the optimization
%procedure to update the online models incrementally. The initial model is
%constructed firstly using the data gathered from random sampling or Design of
%Experiments (DoE). The model is used to guide the selection of the next point
%towards higher performance. Once a new data point is obtained through
%simulation, the model is updated with the new data point. For example, GASPAD
%was proposed in~\cite{liu2014gaspad} for microwave circuit synthesis.

%Multi-objective optimization has also been introduced for automated
%analog circuit design~\cite{MO_overview}, as analog circuits usually have
%several conflicting performances. It would be helpful to provide the designers
%a set of designs with the best trade-off between performances. Unlike the
%aforementioned single-objective optimization algorithms, a multi-objective
%optimization algorithm tries to find a set of \emph{non-dominated} designs that
%approximate the Pareto front of design objectives. For multi-objective analog
%circuits optimization, meta-heuristic based algorithms are commonly used. The
%Non-dominated Sorting Genetic Algorithm (NSGA-II)~\cite{nsgaii}, the
%multi-objective evolutionary algorithm based on decomposition
%(MOEA/D)~\cite{moead}, and the particle swarm intelligence (PSO)
%algorithm~\cite{vural2012analog} with weighted sum aggregation serve as the
%optimization kernel of~\cite{GENOM-POF, AIDA-C, 7927171,LiuBo_MOEAD,
%fakhfakh2010analog}.


%In this paper, we propose a Weighted Expected Improvement based Bayesian
%Optimization (WEIBO) approach for the automated analog circuit sizing. Based on the
%prior belief of the objective functions, the probabilistic surrogate models of
%the objective functions are refined incrementally via Bayesian posterior
%updating when new data are observed. The probabilistic models of the objective
%functions are used to guide the optimization procedure. Gaussian processes (GP)
%are used as the online surrogate models for circuit performances in our
%proposed approach. It is important to balance the \emph{exploration} and
%\emph{exploitation} during the optimization. The \emph{exploration} means that
%the next point tends to explore the unknown regions where the uncertainty in
%the surrogate model is large. The \emph{exploitation} means that the next point
%tends to be the optimum which is predicted by the surrogate model with high
%probability. Acquisition functions are introduced in Bayesian optimization to
%balance the exploration and exploitation. Expected Improvement (EI)
%is selected as the acquisition function in our proposed approach.
%The expected improvement is weighted by the probability of satisfying the
%constraints. The weighted EI (wEI) can well balance the exploration and
%exploitation, and thus the wEI based Bayesian optimization framework can
%significantly reduce the number of simulations while achieving better
%optimization results.

%The Bayesian optimization algorithm is further extended to handle
%multi-objective optimization problems in this paper. We propose to transform
%the multi-objective problems into single-objective problems that can be solved
%by our proposed WEIBO algorithm by Tchebysheff
%scalarization~\cite{MO_overview} with smooth relaxation.

%% \textcolor{blue}{The Bayesian optimization has been applied to the optimization
%% of temperature gradient and skew of 3D IC~\cite{TVLSI-3DIC}.
%% In~\cite{TVLSI-3DIC}, the lower confidence bound (LCB) is selected as the
%% acquisition function, however, the constraints are not considered. For
%% automated analog circuit sizing, GASPAD method~\cite{liu2014gaspad} also employ
%% the Gaussian process based surrogate models to approximate the objective
%% functions.  With the surrogate model, the Differential Evolutionary (DE) like
%% algorithm was used to explore the state space. However, the uncertainty
%% information of design constraints are not well exploited in GASPAD to fully
%% balance the exploitation and exploration. Our experimental results show that
%% our proposed approach can achieve better performances with much less number of
%% simulations compared with GASPAD. This is mainly because we maximize the
%% acquisition function, i.e., weighted expected improvements, to optimally
%% balance the exploration and exploitation.}

%The proposed WEIBO method was tested with several test cases including two
%passive components working at 60GHz, a low power operational amplifier, a power
%amplifier, a voltage controlled oscillator (VCO), and a charge pump. Compared
%with the state-of-the-art approaches listed in the paper, the proposed WEIBO method can achieve
%better optimization results with significantly less number of simulations.

%%\added{A summary of the contributions of this paper is as follows: firstly, we
%%applied Bayesian optimization method to handle problems of analog IC synthesis,
%%although Bayesian optimization has already been used in areas like tuning of
%%neural network hyper-parameters, to the best of our knowledge, it has not been
%%used for analog IC optimization, secondly, we propose an extension to the
%%origin wEI-based Bayesian optimization so that it can handle constrained
%%multi-objective circuit optimization.  }

\documentclass[landscape,paperwidth=200cm, paperheight=120cm,fontscale=0.25, margin=5cm]{baposter}

\usepackage[vlined]{algorithm2e}
\usepackage{times}
\usepackage{calc}
\usepackage{url}
\usepackage{graphicx}
\usepackage{amsmath}
\usepackage{amssymb}
\usepackage{relsize}
\usepackage{multirow}
\usepackage{booktabs}

\usepackage{graphicx}
\usepackage{multicol}
\usepackage[T1]{fontenc}
\usepackage{ae}



%%%%%%%%%%%%%%%%%%%%%%%%%%%%%%%%%%%%%%%%%%%%%%%%%%%%%%%%%%%%%%%%%%%%%%%%%%%%%
%% Begin of Document
%%%%%%%%%%%%%%%%%%%%%%%%%%%%%%%%%%%%%%%%%%%%%%%%%%%%%%%%%%%%%%%%%%%%%%%%%%%%%
\begin{document}
%%%%%%%%%%%%%%%%%%%%%%%%%%%%%%%%%%%%%%%%%%%%%%%%%%%%%%%%%%%%%%%%%%%%%%%%%%%%%
%% Here starts the poster
%%---------------------------------------------------------------------------
%% Format it to your taste with the options
%%%%%%%%%%%%%%%%%%%%%%%%%%%%%%%%%%%%%%%%%%%%%%%%%%%%%%%%%%%%%%%%%%%%%%%%%%%%%
\begin{poster}{
 % Show grid to help with alignment
 grid=true,
 % Column spacing
 colspacing=0.7em,
 % Color style
 headerColorOne=cyan!20!white!90!black,
 borderColor=cyan!30!white!90!black,
 % Format of textbox
 textborder=faded,
 % Format of text header
 headerborder=open,
 headershape=roundedright,
 headershade=plain,
 background=none,
 bgColorOne=cyan!10!white,
 headerheight=0.12\textheight}
 % Eye Catcher
 {
      % \includegraphics[width=0.08\linewidth]{track_frame_00010_06}
      % \includegraphics[width=0.08\linewidth]{track_frame_00450_06}
      % \includegraphics[width=0.08\linewidth]{track_frame_04999_06}
 }
 % Title
 {Batch Bayesian Optimization via Multi-objective Acquisition Ensemble for Automated Analog Circuit Design}
 % Authors
    {
        \vskip 0.15in
        Wenlong Lyu$^1$, Fan Yang$^1$, Changhao Yan$^1$, Dian Zhou$^{1, 2}$ and Xuan Zeng$^1$\\
        $^1$ State Key Lab of ASIC and System, School of Microelectronics, Fudan University, Shanghai, China \\
        $^2$ Department of Electrical Engineering, University of Texas at Dallas, Richardson, TX, U.S.A
    }
 % University logo
 {
  % \begin{tabular}{r}
  %   \includegraphics[height=0.12\textheight]{logo}\\
  %   \raisebox{0em}[0em][0em]{\includegraphics[height=0.03\textheight]{msrlogo}}
  % \end{tabular}
 }

%%%%%%%%%%%%%%%%%%%%%%%%%%%%%%%%%%%%%%%%%%%%%%%%%%%%%%%%%%%%%%%%%%%%%%%%%%%%%%
%%% Now define the boxes that make up the poster
%%%---------------------------------------------------------------------------
%%% Each box has a name and can be placed absolutely or relatively.
%%% The only inconvenience is that you can only specify a relative position 
%%% towards an already declared box. So if you have a box attached to the 
%%% bottom, one to the top and a third one which should be inbetween, you 
%%% have to specify the top and bottom boxes before you specify the middle 
%%% box.
%%%%%%%%%%%%%%%%%%%%%%%%%%%%%%%%%%%%%%%%%%%%%%%%%%%%%%%%%%%%%%%%%%%%%%%%%%%%%%

%%%%%%%%%%%%%%%%%%%%%%%%%%%%%%%%%%%%%%%%%%%%%%%%%%%%%%%%%%%%%%%%%%%%%%%%%%%%%%
  \headerbox{Contribution: Fast \emph{and} reliable Face Alignment}{name=contribution,column=0,row=0,span=2}{
%%%%%%%%%%%%%%%%%%%%%%%%%%%%%%%%%%%%%%%%%%%%%%%%%%%%%%%%%%%%%%%%%%%%%%%%%%%%%%
  Inverse compositional image alignment (ICIA) is fast, but not reliable. We
  explain ICIA from a different perspective which leads naturally to two new
  algorithms with a better capture range and comparable speed.
  }
% %%%%%%%%%%%%%%%%%%%%%%%%%%%%%%%%%%%%%%%%%%%%%%%%%%%%%%%%%%%%%%%%%%%%%%%%%%%%%%
%   \headerbox{There is no \emph{inverse} in \emph{ICIA}}{name=abstract,column=0,below=contribution}{
% %%%%%%%%%%%%%%%%%%%%%%%%%%%%%%%%%%%%%%%%%%%%%%%%%%%%%%%%%%%%%%%%%%%%%%%%%%%%%%
%     Image aligment minimizes
%     \begin{align}
%       \F{\qq} &\defined \normLR{ \e{\qq,\pb} }^2_{\D},\label{eqn:f}\\
%     \text{with }  \e{\qq} &\defined a - I\circ \W{\qq}\nonumber
%     \end{align}
%     composition with an incremental warp $\Vs$ approximates $\Fs$ around $\qq_0$ as
%     \begin{align}
%       \F{\C{\circ}{\qq_0, \pp}} \approx \FF{\qq_0, \pp} & \defined \normLR{\ee{\qq_0,\pp}}^2_{\D}\label{eqn:comp}\\
%       \text{with } \ee{\qq_0, \pp}&\defined  \INTf(a -  I \circ \W{\qq_0} \circ \V{\pp}) \quad.\nonumber
%     \end{align}
%     The gradient descent or Gauss-Newton update rule then gives an estimate of
%     the incremental warp, which drives the model warp.

%     ICIA can be derived by substituting the current backwarped image with the
%     model appearance after taking the derivative. The substitution can be used
%     to get an approximate gradient and/or Hessian, leading to a family of
%     algorithms.

%     Additionally we replace the incremental warp $V$ with an orthonormalized
%     warp and regularize in the composition step. The result is a vast
%     improvement in robustness without sacrificing speed.
% }



%  %%%%%%%%%%%%%%%%%%%%%%%%%%%%%%%%%%%%%%%%%%%%%%%%%%%%%%%%%%%%%%%%%%%%%%%%%%%%%%
%    \headerbox{Our methods are at the performance/speed sweet point}{name=speed,column=2,row=0,span=2}{
%  %%%%%%%%%%%%%%%%%%%%%%%%%%%%%%%%%%%%%%%%%%%%%%%%%%%%%%%%%%%%%%%%%%%%%%%%%%%%%%
%    \newlength{\MSZ}
%    \setlength{\MSZ}{0.01\textwidth}
%    \newcommand{\MarkerCircle}[1]{%
%      \tikz{\draw[use as bounding box] (0,0); \draw[fill=#1]             circle(\MSZ);}}
%    \newcommand{\MarkerRectangle}[1]{%
%      \tikz{\draw[use as bounding box] (0,0); \draw[fill=#1]             (-\MSZ,-\MSZ) rectangle +(2\MSZ,2\MSZ);}}
%    \newcommand{\MarkerDiamond}[1]{%
%      \tikz{\draw[use as bounding box] (0,0); \draw[fill=#1,rotate=45]   rectangle +(2\MSZ,2\MSZ);}}
%    \newcommand{\MarkerTriangle}[1]{%
%      \tikz{\draw[use as bounding box] (0,0); \draw[fill=#1]             (-0.866\MSZ,-0.5\MSZ) -- (0\MSZ,1\MSZ) -- (0.866\MSZ,-0.5\MSZ) -- cycle ;}}
%    \newcommand{\MarkerUDTriangle}[1]{%
%      \tikz{\draw[use as bounding box] (0,0); \draw[fill=#1,rotate=180]  (-0.866\MSZ,-0.5\MSZ) -- (0\MSZ,1\MSZ) -- (0.866\MSZ,-0.5\MSZ) -- cycle ;}}
%    \newcommand{\MarkerPlus}[1]{%
%      \tikz{\draw[use as bounding box] (0,0); \draw[fill=#1]         (-\MSZ,-0.25\MSZ) rectangle +(2\MSZ,0.5\MSZ) (-0.25\MSZ,-\MSZ) rectangle +(0.5\MSZ,2\MSZ);}}
%    \newcommand{\MarkerX}[1]{%
%      \tikz{\draw[use as bounding box] (0,0); \draw[fill=#1,rotate=45]         (-\MSZ,-0.25\MSZ) rectangle +(2\MSZ,0.5\MSZ) (-0.25\MSZ,-\MSZ) rectangle +(0.5\MSZ,2\MSZ);}}
%    \begin{tikzpicture}[x=0.00425\linewidth,y=13mm,font=\smaller]
%      % Ticks
%      \begin{scope}[color=black]
%        \foreach \y in {-0.2218487496,-0.1549019600,-0.0969100130,-0.0457574906,0.0000000000,
%                        0.0000000000,0.3010299957,0.4771212547,0.6020599913,0.6989700043,0.7781512504,0.8450980400,0.9030899870,0.9542425094,1.0000000000,
%                        1.0000000000,1.3010299957,1.4771212547,1.6020599913,1.6989700043,1.7781512504,1.8450980400,1.9030899870,1.9542425094,2.0000000000,
%                        2.0000000000,2.3010299957,2.4771212547,2.6020599913,2.6989700043,2.7781512504,2.8450980400,2.9030899870,2.9542425094,3.0000000000} {
%          \draw[color=lightgray!20!white] (0,\y) -- +(100,0); 
%        }
%        \foreach \y/\lbl in {0/$10^0$,1/$10^1$,2/$10^2$,3/$10^3$} {
%          \draw[color=black]     (0,\y) node[anchor=east,color=black] {\lbl} -- +(100,0); 
%          \draw[color=lightgray] (0,\y) -- +(100,0); 
%          \draw[color=black]     (0,\y) -- +(0.5\MSZ,0mm); 
%        }
%        \foreach \x in {0,20,40,60,80,100} {
%          \draw (\x,-0.23) node[anchor=north,color=black] {\x};
%          \draw[color=lightgray] (\x,-0.23) -- +(0,3.23);
%          \draw[color=black] (\x,-0.23) -- +(0mm,0.5\MSZ);
%        }
%      \end{scope}
%      % Border
%      \draw[color=black] (0,3) -- (0,-0.23)  (100,-0.23) -- (100,3);
%      % Axis labels
%      \draw (50,-0.23)  node[below,yshift=-1em]{Success Rate (\%, Larger is better)};
%      \draw (0,1.5) node[above,rotate=90,yshift=1.6em]{Runtime (smaller is better)};
%      \draw (50,3)  node[above]{The main algorithms starting within $20\%$ \ied{}};
%      % Data
%      \foreach \anch/\rot/\xs/\ys/\x/\y/\ttl/\stl in {
%      west/  0  / 1   / 0   /  5.2631578947   /     0.0000000000 /  Original \ICIA{}        / {\MarkerCircle{blue}},
%    %  west/  0  / 1   / 0   / 14.2369727047   /     0.1128364125 /  \ICIA{} + V^{\text{norm}}/ {\MarkerCircle{blue!70!white}},
%      east/  0  /-1   / 0   / 38.9423076923   /     0.8814699511 /  \CoLiNe{}                / {\MarkerRectangle{green}},
%      west/  0  / 1   / 0   / 39.4696029777   /     0.2610806897 /  \LinCoDe{}               / {\MarkerTriangle{red}},
%      west/  0  / 1   / 0   / 56.8548387097   /     0.9091243545 /  \CoDe{}                  / {\MarkerUDTriangle{yellow}},
%      west/  0  / 1   / 0   / 36.4299007444   /     1.3563671940 /  \CoNe{}                  / {\MarkerX{black!50!white}},
%      west/  0  / 1   / 0   / 41.6918429003   /     2.6132022326 /  L-BFGS (with reg)     / {\MarkerPlus{red!50!blue!50!black}}
%      }{
%        \draw (\x,\y) 
%          node[fill=none,anchor=\anch,xshift=\xs\MSZ,yshift=\ys\MSZ,rotate=\rot] {\ttl} 
%          node{\stl};
%        \draw[fill=black] (\x,\y) circle(0.3\MSZ);
%      }
%    \end{tikzpicture}
%    \begin{tikzpicture}[x=0.00425\linewidth,y=13mm,font=\smaller]
%      % Ticks
%      \begin{scope}[color=black]
%        \foreach \y in {-0.2218487496,-0.1549019600,-0.0969100130,-0.0457574906,0.0000000000,
%                        0.0000000000,0.3010299957,0.4771212547,0.6020599913,0.6989700043,0.7781512504,0.8450980400,0.9030899870,0.9542425094,1.0000000000,
%                        1.0000000000,1.3010299957,1.4771212547,1.6020599913,1.6989700043,1.7781512504,1.8450980400,1.9030899870,1.9542425094,2.0000000000,
%                        2.0000000000,2.3010299957,2.4771212547,2.6020599913,2.6989700043,2.7781512504,2.8450980400,2.9030899870,2.9542425094,3.0000000000} {
%          \draw[color=lightgray!20!white] (0,\y) -- +(100,0); 
%        }
%        \foreach \y/\lbl in {0/$10^0$,1/$10^1$,2/$10^2$,3/$10^3$} {
%          \draw[color=black]     (0,\y) node[anchor=east,color=black] {\lbl} -- +(100,0); 
%          \draw[color=lightgray] (0,\y) -- +(100,0); 
%          \draw[color=black]     (0,\y) -- +(0.5\MSZ,0mm); 
%        }
%        \foreach \x in {0,20,40,60,80,100} {
%          \draw (\x,-0.23) node[anchor=north,color=black] {\x};
%          \draw[color=lightgray] (\x,-0.23) -- +(0,3.23);
%          \draw[color=black] (\x,-0.23) -- +(0mm,0.5\MSZ);
%        }
%      \end{scope}
%      % Border
%      \draw[color=black] (0,3) -- (0,-0.23)  (100,-0.23) -- (100,3);
%      % Axis labels
%      \draw (50,-0.23)  node[below,yshift=-1em]{Success Rate (\%, Larger is better)};
%      \draw (0,1.5) node[above,rotate=90,yshift=1.6em]{Runtime (smaller is better)};
%      \draw (50,3)  node[above,text width=0.5\linewidth,text centered]{All algorithms with $V^{\text{norm}}$ and regularisation};

%      \foreach \anch/\rot/\xs/\ys/\x/\y/\ttl/\stl in {
%        south/0 / 0   /  1.0/       26.4701318852 /         0.6174735535 / \ICIA{} + $V^{\text{norm}}$ / {\MarkerCircle{    blue!50!black}},
%        south/0 / 0   /  1.0/       52.1334367727 /         1.0781366795 / \CoLiNe{}   / {\MarkerRectangle{ green!50!black}},
%        west/ 0 / 1   /  0  /       55.9348332040 /         0.5514820720 / \LinCoDe{}  / {\MarkerTriangle{  red!50!black}},
%        west/ 0 / 1   /  0  /       66.0356865787 /         1.1158139989 / \CoDe{}     / {\MarkerUDTriangle{yellow!50!black}},
%        west/ 0 / 1   /  0  /       66.9511249030 /         1.8260835334 / \CoNe{}     / {\MarkerX{         black!50!white!50!black}},
%        west/ 0 / 1   /  0  /       41.6918429003 /         2.6132022326 / L-BFGS   / {\MarkerPlus{      red!50!blue!50!black}}
%      }{
%        \draw (\x,\y) 
%          node[fill=none,anchor=\anch,xshift=\xs\MSZ,yshift=\ys\MSZ,rotate=\rot] {\ttl} 
%          node{\stl};
%        \draw[fill=black] (\x,\y) circle(0.3\MSZ);
%      }
%    \end{tikzpicture}
%    \begin{multicols}{2}
%    \textbf{Fitting a multiperson AAM. }
%    The best speed--performance tradeoffs come from the two new algorithms
%    \CoDe{} and \LinCoDe{}. Note that \ICIA{} is practically useless on
%    this difficult multi-person dataset with a success rate near zero (left). It
%    can be improved (right) by using the orthonormal incremental warp and
%    regularisation. The \CoDe{} algorithm with regularisation (right) is as
%    accurate as the slow, approximation-free, compositional Gauss-Newton \CoNe{}
%    method but is seven times more efficient.

%    The experiments were performed with leave one identity out on a mixture of two databases (XM2VTS and IMM).
%    \end{multicols}
%    }
% %
% % %%%%%%%%%%%%%%%%%%%%%%%%%%%%%%%%%%%%%%%%%%%%%%%%%%%%%%%%%%%%%%%%%%%%%%%%%%%%%%
% %   \headerbox{Methods Compared}{name=methods,column=0,below=algorithm}{
% % %%%%%%%%%%%%%%%%%%%%%%%%%%%%%%%%%%%%%%%%%%%%%%%%%%%%%%%%%%%%%%%%%%%%%%%%%%%%%%
% %   \begin{tabular}{rllllll}
% %     Method                              & Hessian               &                                        & Gradient        &                                  & Speed      & Capture Range\\
% %     \midrule
% % \CoDe{} (this paper)                & Not used              &                                        & True:           & $\tilde{J}_{\qq_0}^T\e{\qq_0}$   & Fast       & Large  \\[0.1em]
% % \LinCoDe{} (this paper)             & Not used              &                                        & Linear Approx:  & $\bar{J}^T\e{\qq_0}$             & Very Fast  & Medium \\[0.1em]
% % \CoLiNe{}~\cite{burkhardt86:motion} & Constant Approx.:     & $\bar{J}^T\bar{J}$                     & True:           & $\tilde{J}_{\qq_0}^T\e{\qq_0}$   & Fast       & Medium \\[0.1em]
% % \ICIA{}~\cite{matthews:aamr}        & Constant Approx.:     & $\bar{J}^T\bar{J}$                     & Linear Approx:  & $\bar{J}^T\e{\qq_0}$             & Very Fast  & Small  \\[0.1em]
% % \CoNe{}~\cite{matthews:kanade20}    & Gauss-Newton Approx.: & $\tilde{J}_{\qq_0}^T\tilde{J}_{\qq_0}$ & True:           & $\tilde{J}_{\qq_0}^T\e{\qq_0}$   & Slow       & Large  
% %   \end{tabular}
% %   The methods introduced in this paper are Hessian-free gradient descent methods.
% %  }
% %
%  %%%%%%%%%%%%%%%%%%%%%%%%%%%%%%%%%%%%%%%%%%%%%%%%%%%%%%%%%%%%%%%%%%%%%%%%%%%%%%
%    \headerbox{References}{name=references,column=0,above=bottom}{
%  %%%%%%%%%%%%%%%%%%%%%%%%%%%%%%%%%%%%%%%%%%%%%%%%%%%%%%%%%%%%%%%%%%%%%%%%%%%%%%
%      \smaller
     
%      \bibliographystyle{ieee}
%      \renewcommand{\section}[2]{\vskip 0.05em}
%        \begin{thebibliography}{1}\itemsep=-0.01em
%        \setlength{\baselineskip}{0.4em}

%        \bibitem{amberg07:nicp}
%        B.~Amberg, A.~Blake, T.~Vetter
%        \newblock On Compositional Image Alignment with an Application to Activce Appearance Models
%        \newblock In {\em CVPR'09}, 2009.

%        \end{thebibliography}
%    }

%  %%%%%%%%%%%%%%%%%%%%%%%%%%%%%%%%%%%%%%%%%%%%%%%%%%%%%%%%%%%%%%%%%%%%%%%%%%%%%%
%    \headerbox{Training + Testing Data}{name=data,column=0,above=references,below=abstract}{
%  %%%%%%%%%%%%%%%%%%%%%%%%%%%%%%%%%%%%%%%%%%%%%%%%%%%%%%%%%%%%%%%%%%%%%%%%%%%%%%
%    \includegraphics[width=0.2\linewidth]{018_4_2_masked}%
%    \includegraphics[width=0.2\linewidth]{328_2_1_masked}%
%    \includegraphics[width=0.2\linewidth]{319_2_1_masked}%
%    \includegraphics[width=0.2\linewidth]{027_4_2_masked}%
%    \includegraphics[width=0.2\linewidth]{020_1_1_masked}
%    \includegraphics[width=0.2\linewidth]{12_2f_masked}%
%    \includegraphics[width=0.2\linewidth]{21_3m_masked}%
%    \includegraphics[width=0.2\linewidth]{09_6m_masked}%
%    \includegraphics[width=0.2\linewidth]{33_4m_masked}%
%    %\includegraphics[width=0.2\linewidth]{22_3f_masked}
%    $\dots$\\
%    The model was trained from 456 images from the IMM and XM2VTS datasets using
%    120 landmarks. Get the landmarks, model, and source code at:\\
%    \mbox{\url{www.cs.unibas.ch/personen/amberg_brian/aam/}}
%    }

%  %%%%%%%%%%%%%%%%%%%%%%%%%%%%%%%%%%%%%%%%%%%%%%%%%%%%%%%%%%%%%%%%%%%%%%%%%%%%%%

%    \headerbox{Tracking 5000 frames with a general model}{name=tracking,column=2,span=2,below=speed,above=bottom}{
%  %%%%%%%%%%%%%%%%%%%%%%%%%%%%%%%%%%%%%%%%%%%%%%%%%%%%%%%%%%%%%%%%%%%%%%%%%%%%%%
%  {
%  \begin{tabular}{c@{\hspace{0.05em}}c@{\hspace{0.1em}}c@{\hspace{0.1em}}c@{\hspace{0.1em}}c@{\hspace{1em}}c@{\hspace{0.1em}}c@{\hspace{0.1em}}c@{\hspace{0.1em}}c@{\hspace{0.1em}}c}
%    \multicolumn{5}{c}{\smaller \ICIA{} with $\VLins$} &
%    \multicolumn{5}{c}{\smaller \ICIA{} with $\VLins$ + Regularisation}\\[-0.2em]
%    \includegraphics[width=0.095\linewidth]{track_frame_00010_01}&
%    \includegraphics[width=0.095\linewidth]{track_frame_00050_01}&
%    \includegraphics[width=0.095\linewidth]{track_frame_00450_01}&
%    \includegraphics[width=0.095\linewidth]{track_frame_02000_01}&
%    \includegraphics[width=0.095\linewidth]{track_frame_04999_01}&
%    %
%    \includegraphics[width=0.095\linewidth]{track_frame_00010_02}&
%    \includegraphics[width=0.095\linewidth]{track_frame_00050_02}&
%    \includegraphics[width=0.095\linewidth]{track_frame_00450_02}&
%    \includegraphics[width=0.095\linewidth]{track_frame_02000_02}&
%    \includegraphics[width=0.095\linewidth]{track_frame_04999_02}\\[-0.1em]
%    %
%    \multicolumn{5}{c}{\smaller \LinCoDe{}} &
%    \multicolumn{5}{c}{\smaller \LinCoDe{} + Regularisation}\\[-0.2em]
%    \includegraphics[width=0.095\linewidth]{track_frame_00010_03}&
%    \includegraphics[width=0.095\linewidth]{track_frame_00050_03}&
%    \includegraphics[width=0.095\linewidth]{track_frame_00450_03}&
%    \includegraphics[width=0.095\linewidth]{track_frame_02000_03}&
%    \includegraphics[width=0.095\linewidth]{track_frame_04999_03}&
%    %
%    \includegraphics[width=0.095\linewidth]{track_frame_00010_04}&
%    \includegraphics[width=0.095\linewidth]{track_frame_00050_04}&
%    \includegraphics[width=0.095\linewidth]{track_frame_00450_04}&
%    \includegraphics[width=0.095\linewidth]{track_frame_02000_04}&
%    \includegraphics[width=0.095\linewidth]{track_frame_04999_04}\\[-0.1em]
%    %
%    \multicolumn{5}{c}{\smaller \CoDe{}} &
%    \multicolumn{5}{c}{\smaller \CoDe{} + Regularisation}\\[-0.2em]
%    \includegraphics[width=0.095\linewidth]{track_frame_00010_05}&
%    \includegraphics[width=0.095\linewidth]{track_frame_00050_05}&
%    \includegraphics[width=0.095\linewidth]{track_frame_00450_05}&
%    \includegraphics[width=0.095\linewidth]{track_frame_02000_05}&
%    \includegraphics[width=0.095\linewidth]{track_frame_04999_05}&
%    %
%    \includegraphics[width=0.095\linewidth]{track_frame_00010_06}&
%    \includegraphics[width=0.095\linewidth]{track_frame_00050_06}&
%    \includegraphics[width=0.095\linewidth]{track_frame_00450_06}&
%    \includegraphics[width=0.095\linewidth]{track_frame_02000_06}&
%    \includegraphics[width=0.095\linewidth]{track_frame_04999_06}\\[-0.5em]
%    \smaller Frame 10 & \smaller Frame 50 & \smaller Frame 450 & \smaller Frame 2000 & \smaller Frame 5000 &
%    \smaller Frame 10 & \smaller Frame 50 & \smaller Frame 450 & \smaller Frame 2000 & \smaller Frame 5000
%    \end{tabular}
%  }
%    \vspace{-1.2em}
%    \begin{multicols}{2}
%    {\textbf{Our algorithm makes fast and robust tracking possible.}
%      We compare face tracking under natural motion, using \ICIA{},
%      \LinCoDe{} and \CoDe{}. The original \ICIA{} fails
%      immediately with this large model and new face data. Substituting the orthonormal
%      incremental warp for the original \ICIA{} warp, the algorithm still loses track
%      very early, whereas \LinCoDe{} and \CoDe{} can track much
%      further. Finally, adding regularisation to all algorithms, \ICIA{} still
%      loses track completely after approximately 500 frames and does not recover
%      the local deformations accurately. In contrast \CoDe{} now tracks the full
%      5000 frame sequence without reinitialization, and \LinCoDe{} tracks for 2500 frames.}
   
%    The same training dataset was used for both tracking experiments. The
%    training data was aquired with different camera and light settings from
%    different subjects.
%    \end{multicols}
%    }
%  %%%%%%%%%%%%%%%%%%%%%%%%%%%%%%%%%%%%%%%%%%%%%%%%%%%%%%%%%%%%%%%%%%%%%%%%%%%%%%
%    \headerbox{Low Res Tracking}{name=lowrestracking,column=1,span=1,below=speed,above=bottom}{
%  %%%%%%%%%%%%%%%%%%%%%%%%%%%%%%%%%%%%%%%%%%%%%%%%%%%%%%%%%%%%%%%%%%%%%%%%%%%%%%
% \begin{tabular}{@{}c@{}c@{}c@{}c@{}c@{}}
%   \multicolumn{5}{c}{\smaller \ICIA{} with $\VLins$}\\[-0.2em]
%   \includegraphics[width=0.2\linewidth]{bush_00010_02}&
%   \includegraphics[width=0.2\linewidth]{bush_00100_02}&
%   \includegraphics[width=0.2\linewidth]{bush_00200_02}&
%   \includegraphics[width=0.2\linewidth]{bush_00300_02}&
%   \includegraphics[width=0.2\linewidth]{bush_00400_02}\\[-0.1em]
%   \multicolumn{5}{c}{\smaller \LinCoDe{}}\\[-0.2em]
%   \includegraphics[width=0.2\linewidth]{bush_00010_05}&
%   \includegraphics[width=0.2\linewidth]{bush_00100_05}&
%   \includegraphics[width=0.2\linewidth]{bush_00200_05}&
%   \includegraphics[width=0.2\linewidth]{bush_00300_05}&
%   \includegraphics[width=0.2\linewidth]{bush_00400_05}\\[-0.1em]
%   \multicolumn{5}{c}{\smaller \CoDe{}}\\[-0.2em]
%   \includegraphics[width=0.2\linewidth]{bush_00010_08}&
%   \includegraphics[width=0.2\linewidth]{bush_00100_08}&
%   \includegraphics[width=0.2\linewidth]{bush_00200_08}&
%   \includegraphics[width=0.2\linewidth]{bush_00300_08}&
%   \includegraphics[width=0.2\linewidth]{bush_00400_08}\\[-0.5em]
%   \smaller Frame 10 & \smaller Frame 100 & \smaller Frame 200 & \smaller Frame 300 & \smaller Frame 400 
%   \end{tabular}

%   \vspace{1.25em}
%   \textbf{Tracking a low resolution video with large head motions
%   succeeds with \CoDe{}, where \ICIA{} fails.}\\ All methods used the orthonormal
%   incremental warp, and relatively strong regularisation.  \ICIA{} starts to
%   drift in the early frames, while~\CoDe{} tracks the full sequence. The
%   approximate gradient method \LinCoDe{} also suceeds, but looses
%   track of the details for about 100 frames.
%    }

%  %%%%%%%%%%%%%%%%%%%%%%%%%%%%%%%%%%%%%%%%%%%%%%%%%%%%%%%%%%%%%%%%%%%%%%%%%%%%%%%
%   \headerbox{Compositional Alignment}{name=algorithm,column=1,above=lowrestracking,below=contribution}{
%  %%%%%%%%%%%%%%%%%%%%%%%%%%%%%%%%%%%%%%%%%%%%%%%%%%%%%%%%%%%%%%%%%%%%%%%%%%%%%%%
%   \begin{algorithm}[H]
%     \dontprintsemicolon
%     \linesnumbered
%     \For{Blur and regularisation values}{
%       \nl Initialize $\qq, \qq_{\text{best}}$ and $\kappa$\;
%       \Repeat{converged}{
%         \nl Calculate $\Nabla{\pp}{\tilde{F}(\qq,\VEC 0)}$, $F(\qq)$\;
%         \eIf{$F(\qq) < F(\qq_{\text{best}})$}{
%           %\nl Calculate distance between best warp estimate and current warp estimate to test for convergence\;
%           \nl $\qq_{\text{best}} \gets \qq$\;
%           %\If{More than three successiv updates}{ (Too much detail)
%             \nl Increase $\kappa$\;
%           %}
%         }{
%           \If{$\kappa$ smaller than threshold}{
%             \nl return\;
%           }%{
%              decrease $\kappa$\;
%           %}
%         }
%         \nl Calculate $\pp$ from $\Nabla{\pp}{\tilde{F}(\qq_{best},\pp)}$ and $\kappa$\;
%         \nl $\qq \gets \C{\circ}{\qq, \pp}$
%       }
%     }
%   \end{algorithm}
%    }
\end{poster}%
%
\end{document}
